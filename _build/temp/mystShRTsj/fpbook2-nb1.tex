\section{Waarden en functies}

(Inleiding)

\subsection{Waarden, namen en expressies}

\begin{framed}
\textbf{Begrippen}\\
(enkelvoudige) waarde, operator, expressie, prioriteit (of ``sterkte van binding''), naam (-definitie, -gebruik)
\end{framed}

Als eerste stap in de kennismaking met functioneel programmeren beginnen we erg eenvoudig: met \textit{waarden}, zoals gehele getallen, en \textit{expressies} waarin je (invoer)waarden gebruikt om nieuwe (uitvoer)waarden te berekenen.

Een \textbf{enkelvoudige waarde} is een \textit{getal} of een \textit{teken} (\textit{character}). Voorlopig gebruiken we in de voorbeelden alleen \textit{gehele getallen}; maar zwevende komma getallen (\textit{floating point}) zijn ook mogelijk in Haskell.

Later zullen we kennismaken met \textit{samengestelde waarden}, zoals strings en lijsten.

In \textbf{expressies} kunnen we waarden combineren met \textit{operatoren} en \textit{functies}, om daarmee nieuwe waarden uit te rekenen. De gebruikelijke operatoren zijn: \texttt{+ - *} voor optellen, aftrekken en vermenigvuldigen. Voor het delen van gehele getallen hebben we de functies \texttt{div} en \texttt{mod}. Vermenigvuldigen en delen hebben een grotere prioriteit dan optellen en aftrekken, alweer zoals we gewend zijn. We kunnen haakjes gebruiken om de prioriteit (volgorde van berekenen) expliciet aan te geven.

(Operatoren blijken in Haskell ook gewoon functies te zijn; daar komen we later op terug.)

Voorbeelden van expressies

\begin{verbatim}
3 + 4 * 5
\end{verbatim}

\begin{verbatim}
23
\end{verbatim}

\begin{verbatim}
(3 + 4) * 5
\end{verbatim}

\begin{verbatim}
35
\end{verbatim}

\textbf{Opdracht.} Geef voor bovenstaande expressies aan wat de waarden zijn en wat de operatoren.

\subsection{Namen: definitie en gebruik}

We kunnen een waarde een \textbf{naam} geven (\textit{definiëren}), en die naam later \textit{gebruiken} waar we de waarde nodig hebben. Bij het uitrekenen van een expressie vervangen we een naam dan door de waarde die aan die naam gekoppeld is.

In een functionele taal als Haskell mag een naam precies één keer gedefinieerd zijn: je kunt in eenzelfde programma die naam geen andere waarde geven. \texttt{a} is dus geen variabele in de zin van ``gewone'' programmeertalen, het is meer zoals een naam in de wiskunde die een enkele waarde voorstelt.

\textbf{let op:} namen van waarden en van functies moeten in Haskell met een kleine letter beginnen.

\begin{verbatim}
a = 3   - - definitie van a
\end{verbatim}

\begin{verbatim}
a + 4 * 5  - - gebruik van a
\end{verbatim}

\begin{verbatim}
23
\end{verbatim}

\begin{verbatim}
b = 4 * 5 - - definitie van b
\end{verbatim}

\begin{verbatim}
a + b  - - gebruik van a en b
\end{verbatim}

\begin{verbatim}
23
\end{verbatim}

\textbf{Opmerking.} Wanneer reken je de expressie in de definitie van \texttt{b} uit? In onze voorbeelden gaan we ervan uit dat je eerst de expressie uitrekent, en de resultaat-waarde gebruikt als definitie van \texttt{b}. Maar: in Haskell maakt dat niet uit, omdat je een naam maar éénmaal een waarde mag geven: je zou de berekening ook later kunnen uitvoeren (\textit{lazy evaluation}).

Een expressie beschrijft een berekening. Deze berekening kun je met de hand uitvoeren, waarbij je elke stap uitlegt wat je doet.

Voorbeeld:

\begin{verbatim}
2 * a + b
==     {vervang a door de bijbehorende waarde: 3}
    2 * 3 + b
==     {rekenen}
    6 + b
==     {vervang b door 20}
    6 + 20
==     {rekenen}
    26
\end{verbatim}

Dit blijkt te kloppen met het Haskell-resultaat

\begin{verbatim}
2 * a + b
\end{verbatim}

\begin{verbatim}
26
\end{verbatim}

Bovenstaande uitwerking lijkt erg omslachtig, maar we zullen later voorbeelden zien waarin zulke berekeningen kunnen helpen om programma's te begrijpen.

\subsection{Functies: definitie en gebruik}

In de voorbeelden hierboven staan de namen \texttt{a} en \texttt{b} voor enkelvoudige waarden (gehele getallen).
We kunnen een naam ook koppelen aan een \textit{functie}.

\begin{itemize}
\item de definitie van een functie koppelt de functie-naam aan een functie-waarde
\item het gebruik van een functie(-naam) met een argument-waarde betekent de toepassing van de functie op die waarde.
\end{itemize}

\textbf{functie-toepassing.} In een expressie kunnen we ook functies gebruiken: we spreken dan functie-toepassing (functie-applicatie) of functie-aanroep. De functie \texttt{abs} \textit{toegepast op het argument} \texttt{5} schrijf je in Haskell als: \texttt{abs 5} - de naam van de functie met daarna het argument.

In de wiskunde en in veel programmeertalen zou je schrijven: \texttt{abs(5)}, maar in Haskell gebruik je voor functie-toepassing geen haakjes.

Je hebt soms wel haakjes nodig om de rekenvolgorde te beschrijven (prioriteit of binding): functie-toepassing ``bindt sterker'' dan operatoren zoals \texttt{+} en \texttt{*}. \texttt{abs a + b} betekent dan: \texttt{(abs a) + b}, als je het anders bedoelt moet je dit schrijven als: \texttt{abs (a+b)},

\begin{verbatim}
abs (a - 10)
\end{verbatim}

\begin{verbatim}
7
\end{verbatim}

\begin{verbatim}
abs a - 10
\end{verbatim}

\begin{verbatim}
-7
\end{verbatim}

\textbf{functie-definitie.} Als eerste voorbeeld van een functie definiëren we de functie \texttt{sqr}, voor het kwadrateren van een getal. De gebruikerlijke wiskundige definitie is:

\begin{equation}
f(x) = x * x
\end{equation}

In Haskell wordt dit:

\begin{verbatim}
sqr x = x * x
\end{verbatim}

Deze regel definieert de naam \texttt{sqr} als de naam van de functie, met één \textit{parameter}, die we hier \texttt{x} noemen. De definitie koppeld de naam \texttt{sqr} aan de geparametriseerde expressie, \texttt{x * x}.

De \textit{naam} van de parameter \texttt{x} heeft alleen betekenis in de functie-definitie. We kunnen deze definitie ook schrijven als: \texttt{sqr y = y * y} of als \texttt{sqr abc = abc * abc}.

Merk op dat we hier, net als bij functie-toepassing, geen haakjes schrijven.

De toepassing van deze functie op een argument, bijvoorbeeld \texttt{sqr 7}, resulteert in de expressie (berekening) waarin de waarde van het afgument ingevuld is in de definiërende expressie van de functie, hier \texttt{sqr}:

\begin{verbatim}
sqr 7
==    {functie -toepassing sqr: vervang de naam door de definiërende expressie}
    let x = 7 in x * x
==    {vervang x door waarde: 7}
    7 * 7
==    {rekenen}
    49
\end{verbatim}

Met de \texttt{let}-constructie in Haskell kun je een lokale naam definiëren, die buiten deze \texttt{let} geen betekenis heeft. Op die manier voorkomen je naamsconflicten.

\begin{verbatim}
let z = 12 in z + z
\end{verbatim}

\begin{verbatim}
24
\end{verbatim}

De berekening in Haskell geeft hetzelde resultaat als hierboven:

\begin{verbatim}
sqr 7
\end{verbatim}

\begin{verbatim}
49
\end{verbatim}

\begin{framed}
\textbf{Begrippen}\\
\begin{itemize}
\item functie-definitie (vs. functie-toepassing of -gebruik)
\item functienaam
\item (formele) parameter - als lokale naam
\item (geen haakjes)
\item scope van een lokale naam; ``uniek'' ten opzichte van andere namen in programma
\item ``body'' - of expressie? - waarin deze parameter voorkomt, ``geparametriseerd''
\end{itemize}
\end{framed}

We kunnen deze functie \textit{toepassen} op het \textit{argument} 3. Dit is een voorbeeld van functie-applicatie ofwel functie-toepassing.

\begin{verbatim}
sqr 3
\end{verbatim}

\begin{verbatim}
9
\end{verbatim}

Mogelijke voorbeelden:

\begin{itemize}
\item double
\item succ
\item isEven
\item (omzetten van hoofd- in kleine letters of omgekeerd)
\end{itemize}

\textbf{Voorbeeld.} In het onderstaande voorbeeld gebruiken we een \textit{expressie} als argument voor de sqr-functie. In dit geval moeten we haakjes gebruiken voor deze expressie, omdat functie-applicatie sterker ``bindt'' dan optelling (of andere operatoren). De notatie \texttt{sqr a + 2} moet je dan lezen als \texttt{(sqr a) + 2}. \textit{Ga dit na.}

\begin{verbatim}
sqr (a + 2)
\end{verbatim}

\begin{verbatim}
25
\end{verbatim}

We rekenen deze expressie uit door \textit{eerst de argument-expressie \texttt{a + 2} uit te rekenen}:

\begin{verbatim}
sqr (a + 2)
==     {invullen waarde van a}
    sqr (3 + 2)
==     {rekenen}
    sqr 5
==     {invullen definitie van sqr}
    let x = 5 in x * x
==     {invullen waarde van x}
    5 * 5
==     {rekenen}
    25
\end{verbatim}

We kunnen de volgorde van de verschillende stappen ook anders kiezen, bijvoorbeeld:

\begin{verbatim}
sqr (a + 2)
== - - invullen definitie van sqr
    let x = a + 2 in x * x
== - - invullen van waarde van x
    (a + 2) * (a + 2)
== - - invullen van waarde van a
    (3 + 2) * ( 3 + 2)
== - - rekenen
    5 * 5
== - - rekeken
    25
\end{verbatim}

Merk op dat we hetzelde resultaat krijgen als eerder. In een functionele taal maakt de volgorde van de stappen geen verschil voor het resultaat, omdat de waarde bij een naam nooit verandert. Maar deze volgorde kan wel verschil maken voor de \textit{hoeveelheid rekenwerk}. In dit laatste voorbeeld moesten we bijvoorbeeld de uitdrukking \texttt{3 + 2} tweemaal uitrekenen.

In het vervolg hanteren we steeds de regel: eerst de argumenten uitreken, daarna de functie-definitie invullen. Deze manier van het gebruik van functies heet \textit{strict evaluation}. Dit is de aanpak die de meeste programmeertalen gebruiken, ook veel functionele talen.

Haskell zelf gebruikt een andere aanpak: een expressie wordt pas uitgerekend als het resultaat nodig is, bijvoorbeeld omdat dit afgedrukt moet worden. Deze aanpak heet \textit{lazy evaluation}. We zullen daar later voorbeelden van laten zien, en gevallen waar dit extra gemak geeft.

\subsubsection{Functies met meerdere parameters}

\begin{verbatim}
add x y = x + y
\end{verbatim}

\begin{verbatim}
add 3 (4*5)
\end{verbatim}

\begin{verbatim}
23
\end{verbatim}

Merk op: geen haakjes, geen komma. Maar wel haakjes voor de expressie \texttt{4*5}, omdat \textit{functie-applicatie sterker bindt dan vermenigvuldigen}. Zonder deze haakjes zou de betekenis zijn: \texttt{(add 3 4) * 5} (\textit{controleer dit})
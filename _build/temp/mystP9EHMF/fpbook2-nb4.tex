\section{Functies als waarden}

In dit hoofdstuk maak je kennis met functies als waarden. Een functie kan overal gebruikt worden waar je een ``normale'' waarde zoals een getal of een string kunt gebruiken:

\begin{itemize}
\item een functie kan optreden als \textit{parameter} van een andere functie;
\item een functie kan optreden als \textit{resultaat} van een functie;
\item een functie kan onderdeel zijn van een samengestelde datastructuur, zoals een lijst;
\item je kunt ``rekenen met functies'', bijvoorbeeld door de \textit{compositie van twee functies}.
\end{itemize}

Je kunt ook anonieme functie-waarden kunt hebben: een functie-waarde hoeft niet beslist een naam te hebben, net zomin als elke getalwaarde of string een naam hoeft te hebben.

\subsection{Lambda-expressies: anonieme functie-waarden}

Tot nu toe hebben we functie-definities gezien van de vorm: \texttt{sqr x = x * x}. Dit lijkt nog niet op de vorm van andere naam-waarde koppelingen, zoals \texttt{a = 42}. De waarde \texttt{42} is hier een anonieme waarde, die in deze definitie gekoppeld wordt aan een naam.

Een lambda-expressie beschrijft een \textit{anonieme functie}: deze bestaat alleen uit de parameters en de definiërende expressie. Bijvoorbeeld:

\begin{verbatim}
\ x -> x * x
\end{verbatim}

Het symbool \texttt{{\textbackslash}} spreek je uit als \textit{lambda}, zie ook de gelijkenis met de griekse letter $\lambda$. Het pijltje \texttt{-\textgreater } kun je uitspreken als \texttt{naar} of \texttt{geeft}. Merk de overeenkomst op tussen deze lambda-expressie en de notatie voor een functie-type.

Wat kun je met zo'n anonieme functie-expressie (of eigenlijk: functie-waarde)? Deze kun je gebruiken overal waar je een functie-waarde verwacht: in functie-definities, parameters of resultaten, enz.

Hiermee kunnen we de definitie van een functie schrijven op dezelfde manier als een ``normale'' naam-waarde definitie:

\begin{verbatim}
sqr :: Int -> Int
sqr = \ x -> x * x
\end{verbatim}

De toepassing van een functie op een argument kun zien als het vervangen van de \texttt{{\textbackslash}} door \texttt{let}, met de parameter-naam gekoppeld aan het argument; en het pijltje \texttt{-\textgreater } vervangen door \texttt{in}

\begin{verbatim}
sqr 7
==    {invullen waarde van sqr}
    (\ x -> x * x) 7
==    {functie -toepassing op argument 7)
    let x = 7 in x * x
==    {invullen waarde van  x}
    7 * 7
==    {rekenen}
    49
\end{verbatim}

\begin{verbatim}
sqr :: Int -> Int
sqr = \ x -> x * x
\end{verbatim}

\begin{verbatim}
sqr 7
\end{verbatim}

\begin{verbatim}
49
\end{verbatim}

\begin{verbatim}
(\ y -> y * y) 7
\end{verbatim}

\begin{verbatim}
49
\end{verbatim}

\textbf{Opdracht} Definieer de functie \texttt{double} met een lambda-expressie.

\textbf{functies en lambda-expressies met meerdere parameters}

Een lambda-expressie kan ook meerdere parameters hebben:

\begin{verbatim}
add :: Int -> Int -> Int
add = \ x y -> x + y
\end{verbatim}

\begin{verbatim}
add :: Int -> Int -> Int
add = \ x y -> x + y
\end{verbatim}

\begin{verbatim}
add 3 4
\end{verbatim}

\begin{verbatim}
7
\end{verbatim}

\subsection{Functie als operator, operator als functie}

Een functie met twee parameters, zoals \texttt{add} hierboven, kun je ook als (binaire) operator \textit{tussen de parameters} noteren, zoals een \texttt{+} of \texttt{*} operator. Je gebruikt daarvoor ``backticks'' (```) rond de naam.

\begin{verbatim}
3 `add` 4 `add` 5
\end{verbatim}

Alleen de notatie verandert, het principe van functie-aanroep blijft gelijk.

\begin{verbatim}
3 `add` 4 `add` 5
\end{verbatim}

\begin{verbatim}
12
\end{verbatim}

\textbf{Operatoren als functie.} Je kunt een operator, zoals \texttt{+} ook als functie gebruiken, door deze tussen haakjes te plaatsen: \texttt{(+)}. bijvoorbeeld: \texttt{(+) 3 4}

\begin{verbatim}
(+) 3 4
\end{verbatim}

Bij ingewikkelder expressies moet je er dan rekening mee houden dat functie-toepassing ``links-associatief'' is. In de onderstaande expressie is daarom een extra haakjes-paar nodig voor de vermenigvuldiging.

\begin{verbatim}
(+) 3 ((*) 4 5)
\end{verbatim}

\begin{verbatim}
23
\end{verbatim}

\subsection{Functie als parameter: map, reduce, filter}

Soms wil je een functie toepassen op alle waarden in een datastructuu, zoals een lijst.
Hiervoor heb je in Haskell (en andere talen) de functie \texttt{map} die zowel een functie als een lijst als parameter heeft: deze \texttt{map} past de betreffende functie toe op alle elementen van de lijst:

\texttt{map f [a, b, c, ...] = [f a, f b, fc, ...]}

Het type van \texttt{map} voor \texttt{Int}-lijsten kun je dan zien als: \texttt{map :: (Int -\textgreater  Int) -\textgreater  [Int] -\textgreater  [Int]}
De eerste parameter is een functie, de tweede is een Int-lijst, het resultaat is ook weer een Int-lijst.

\begin{verbatim}
sqr :: Int -> Int
sqr x = x * x
\end{verbatim}

Voorbeeld

\begin{verbatim}
map sqr [1,3,7]
==    {"definitie" map invullen}
    [sqr 1, sqr 3, sqr 7]
==    {definitie sqr invullen}
    [let x=1 in x*x, let x=3 in x*x, let x=7 in x*x]
==    {definitie (waarde) van x invullen}
    [1*1, 3*3, 7*7]
==    {rekenen}
    [1, 9, 49]
\end{verbatim}

\begin{verbatim}
map sqr [1, 3, 7]
\end{verbatim}

\begin{verbatim}
[1,9,49]
\end{verbatim}

\begin{verbatim}
double :: Int -> Int
double x = x + x
\end{verbatim}

\begin{verbatim}
map double [1..10]
\end{verbatim}

\begin{verbatim}
[2,4,6,8,10,12,14,16,18,20]
\end{verbatim}

Je kunt ook een \textit{anonieme functie} (lambda expressie) meegeven aan \texttt{map}:

\begin{verbatim}
map (\x -> x * x * x) [1..10]
\end{verbatim}

\begin{verbatim}
[1,8,27,64,125,216,343,512,729,1000]
\end{verbatim}

\textbf{foldl, foldr} Een andere veel voorkomende operatie is om een functie \textit{tussen} alle elementen van een lijst uit te voeren, bijvoorbeeld het optellen van alle elementen.

\texttt{[1, 3, 7]} -\textgreater  \texttt{1 + 3 + 7} -\textgreater  \texttt{11}

Als de lijst leeg is, is het resultaat 0 (voor optelling). Eigenlijk kun je bovenstaande dan schrijven als:

\texttt{[1, 3, 7]} -\textgreater  \texttt{1 + 3 + 7 + 0} -\textgreater  \texttt{11}

(Wat is de waarde die je moet gebruiken voor het vermenigvuldigen van alle elementen van een lijst? Anders gezegd: welke bijdrage levert de lege lijst aan de totale vermenigvuldiging?)

De functie \texttt{foldl} (\textit{fold left}) in Haskell past een operator \texttt{f} (functie met 2 parameters) toe \textit{tussen} alle elementen van een lijst, waarbij het resultaat voor de lege lijst gelijk is aan de ``nul-waarde'' \texttt{z}:

\begin{verbatim}
foldl :: (Int -> Int) -> Int -> [Int] - Int
foldl f z [a, b, c, ...] = a `f` b `f` c `f` ... `f` z
\end{verbatim}

Dit kun je ook zien als het vervangen van de const-constructor \texttt{:} door de functie (operator) \texttt{f}, waarbij de waarde voor de lege lijst gelijk is aan de ``nul-waarde'' \texttt{z}:

\begin{verbatim}
foldl f z (a : b : c ... : []) = a `f` b `f` c ... `f` z
\end{verbatim}

Voorbeeld:

\begin{verbatim}
foldl (+) 0 (1 : 3 : 7 : []) = 1 + 3 + 7 + 0 = 11
\end{verbatim}

De functie \texttt{foldl} vouwt de lijst wordt eigenlijk ``opgevouwen'' tot een enkele waarde met behulp van de functie \texttt{f}, vandaar de naam \texttt{foldl} (fold left).

Naast \texttt{foldl} heb je ook \texttt{foldr} (\textit{fold right}). Het verschil is dat \texttt{foldl} van links naar rechts rekent (links-associatief), en foldr van rechts naar links (rechts-associatief). Voor het optellen van gehele getallen maakt dat geen verschil, maar bij andere functies kan dat een ander resultaat geven.

\begin{verbatim}
foldl f z (a : b : c ... : []) = ((((a `f` b) `f` c) ...) `f` z)
foldr f z (a : b : c ... : []) = (a `f` (b `f` (c `f` (... `f` z))))
\end{verbatim}

\textbf{reduce} In andere programmeertalen heet de functie \texttt{foldl} vaak \textit{reduce}: deze reduceert een lijst, of een andere datastructuur, tot een enkele waarde.

De combinatie \textbf{map/reduce} (en soms ook \textit{filter}) komt tegenwoordig in veel programmeertalen voor: dit maakt het mogelijk om potentieel parallellisme uit te drukken, omdat de volgorde van de afzonderlijke operaties er niet toe doet.

\begin{verbatim}
foldl (\x y -> x + y) 0 [1..10]
\end{verbatim}

\begin{verbatim}
55
\end{verbatim}

\begin{verbatim}
foldl (\x y -> x + y) 0 []
\end{verbatim}

\begin{verbatim}
0
\end{verbatim}

\textbf{Operatoren als functie.} Je kunt een operator, zoals \texttt{+} ook als functie gebruiken, door deze tussen haakjes te plaatsen: \texttt{(+)}. Bovenstaand voorbeeld kun je dan ook schrijven als:

\subsection{Functie als resultaat}

Je kunt een functie-waarde ook gebruiken als resultaat van een functie.
Bestudeer het volgende voorbeeld:

\begin{verbatim}
addx :: Int -> (Int -> Int)
addx x = \ y -> x + y

addtwo = addx 2
...
addtwo 9
\end{verbatim}

De functie-toepassing \texttt{addx 2} heeft als resultaat de functie-waarde \texttt{{\textbackslash}~y -\textgreater  2 + y}

\begin{verbatim}
addx 2
\end{verbatim}

==    \{definitie van addx invullen\}
let x = 2 in {\textbackslash}~y -\textgreater  x + y
==    \{definitie van x invullen\}
{\textbackslash}~y -\textgreater  2 + y

We kunnen de definitie van \texttt{addx} ook schrijven als geneste lambda-expressies:

\begin{verbatim}
addx :: Int -> (Int -> Int)
addx = \ x -> (\ y -> x + y)
\end{verbatim}

De haakjes rond het resultaat-type en de haakjes rond de geneste lambda-expressie mogen we weglaten, omdat de pijl \texttt{-\textgreater } \textit{rechts-associatief} is:

\begin{verbatim}
addx :: Int -> Int -> Int
addx = \ x -> \ y -> x + y
\end{verbatim}

De functie \texttt{addx} is nu eigenlijk een gewone functie met twee parameters. Als je alleen het eerste argument opgeeft, krijg je een functie die je later op het tweede argument kunt toepassen.

Alle functies met twee of meer parameters kunnen we beschouwen als \textit{geneste lambda-expressies}. We kunnen de functies \textit{partieel parametriseren}, door deze toe te passen op 1 argument: we krijgen dan een functie die we later op de volgende argumenten kunnen toepassen.

Het verschijnsel dat je functies met meerdere parameters kunt beschouwen als geneste lambda-expressies, die je partieel kunt parametriseren, heet Currying, naar de logicus Haskell (!) Curry. Je mag die dus nu bij zijn voornaam noemen.

\begin{verbatim}
addx :: Int -> (Int -> Int)
addx x = \ y -> x + y
\end{verbatim}

\begin{verbatim}
addtwo = addx 2
\end{verbatim}

\begin{verbatim}
addtwo 3
\end{verbatim}

\begin{verbatim}
5
\end{verbatim}

\begin{verbatim}
addx 2 3
\end{verbatim}

\begin{verbatim}
5
\end{verbatim}

Deze Currying kunnen we toepassen op alle functies en operatoren met twee of meer parameters, zoals ook bijvoorbeeld de normale reken-operatoren:

\begin{verbatim}
succ = (+) 1
\end{verbatim}

\begin{verbatim}
succ 4
\end{verbatim}

\begin{verbatim}
5
\end{verbatim}

We kunnen dit ook gebruiken om gespecialiseerde lijst-functies te maken, op basis van algemene lijstfuncties zoals \texttt{map} of \texttt{foldl}.

\begin{verbatim}
sum = foldl (+) 0
\end{verbatim}

\begin{verbatim}
sum [1..10]
\end{verbatim}

\begin{verbatim}
55
\end{verbatim}

\subsection{Functies in datastructuren}

lijst met functies.
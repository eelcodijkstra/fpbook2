\section{Een voorproefje van functioneel programmeren in Haskell}

Je kunt Haskell als een rekenmachine gebruiken: je kunt expressies invoeren in de gebruikelijke notatie.

Denk erom dat je, zoals in alle programmeertalen, vermenigvuldiging altijd moet uitschrijven met de \texttt{*}=operator.

\begin{verbatim}
3 + 4
\end{verbatim}

\begin{verbatim}
7
\end{verbatim}

\begin{verbatim}
3 + 4 * 5
\end{verbatim}

\begin{verbatim}
(3 + 4) * 5
\end{verbatim}

Je kunt een waarde ook een naam geven, en die naam gebruiken op de plek van een waarde in een expressie:

\begin{verbatim}
a = 3
\end{verbatim}

\begin{verbatim}
a + 4
\end{verbatim}

Naast getallen als waarde kun je ook tekens (\textit{chars}) en \textit{strings} gebruiken. (Later)

Een functie-definitie ziet er vrijwel net zo uit als in de wiskunde:

\begin{verbatim}
sqr x = x * x
\end{verbatim}

Als je nu de waarde van \texttt{sqr} opvraagt, krijg een foutmelding: je kunt een functie-waarde niet afdrukken.

\begin{verbatim}
sqr
\end{verbatim}

De aanroep (de ``applicatie'' ofwel het gebruik) van een functie schrijf je als de naam van de functie direct gevolgd door het argument. Je hebt geen haakjes nodig.

\begin{verbatim}
sqr 3
\end{verbatim}
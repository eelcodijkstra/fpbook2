\section{Voor de docent}

\begin{itemize}
\item \textbf{PRIMM} Als didactische aanpak gebruiken we, waar mogelijk, de PRIMM methode: Predict, Read, Investigate, Modify, Make. Leerlingen beginnen met het lezen en begrijpen van bestaande code.

\begin{itemize}
\item het is dan belangrijk om goede voorbeelden te kiezen.
\end{itemize}


\item \textbf{Gebruik voor implementatie.} In het algemeen proberen we voorbeelden van het gebruik te laten zien, voordat we ingaan op de implementatie.
\item \begin{itemize}
\item \textbf{Uitwerken met de hand.} Om een goed begrip te krijgen van de verschillende constructies, is het belangrijk om een aantal voorbeelden met de hand uit te werken.
\end{itemize}


\item \textbf{Kleine stappen, controleren cq. testen.} We gebruiken Jupyter Notebook (of -Lab) om programma's op te kunnen bouwen in kleine stappen, waarbij elke stap getest kan worden met een of twee kleine voorbeelden.

\begin{itemize}
\item Welke voorbeelden kies je om te testen? Zorg in elk geval dat je de randgevallen apart test, zoals bijvoorbeeld de lege lijst. En daarnaast een of twee ``normale'' gevallen.
\end{itemize}
\end{itemize}

\subsection{Tips voor het gebruik van Jupyter Notebook}

\begin{itemize}
\item gebruik Jupyter Notebook om te experimenteren met de voorbeelden: pas de voorbeeld-data aan, en pas de voorbeeld-functies aan.
\item de volgorde van de berekening is belangrijk, hiermee bouw je een historie op: na een aantal experimenten kan het zijn dat de actuele inhoud van de cellen niet meer klopt met de opgebouwde berekening-historie (van definities e.d.). Het is handig om regelmatig de kernel opnieuw te starten (bijv. \textit{Restart Kernel and Clear...} of \textit{Restart Kernel and Run up to Selected Cell}).
\item
\end{itemize}